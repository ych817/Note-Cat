\let\ph=\phantom
\let\hph=\hphantom
\let\vph=\vphantom
\let\mrel=\mathrel
\let\mbin=\mathbin
\let\mllap=\mathllap
\let\mrlap=\mathrlap
\let\mclap=\mathclap

\newcommand{\mathHL}[2]{{%                   % 高亮
  \setlength{\fboxsep}{0pt}%
  \colorbox{#1}{\(#2\)}%
}}
\newcommand{\mathHLf}[2]{{%                  % 高亮 (带边框)
  \setlength{\fboxrule}{0.2pt}%
  \setlength{\fboxsep}{0pt}%
  \fcolorbox{#1}{\(#2\)}%
}}

\def\vts#1{\lvert#1\rvert}                     % 竖线 ||
\def\prs#1{\left(#1\right)}                    % 括号 ()
\def\bcs#1{\left\{#1\right\}}                  % 括号 {}
\def\bks#1{\left[#1\right]}                    % 括号 []
\def\plr{\vph{(fg)}}                           % 柱子, 用于指定盒子的最小高度
\def\etc{\textrm{etc}}                         % 对应省略号
\def\wld{\_}                                   % 通配符

% 值构造器
\def\cons{\mbin{.}}                            % 对应积类型 , 同 ","
\def\ethr{\mbin{\vert}}                        % 对应和类型 , 同 "|"

% 几个类型 : 对象 , 范畴 , 态射 , 函子 , 自然变换
\def\Series#1#2#3{%                            % 大类, 用于加高亮色
  \mathHL{#1! #2}%
  {\plr\smash{#3}}
}
\def\objSeries#1#2{\Series{blue}{#1}{\sf #2}}  % 对象类型 , 蓝色底色 , sf 字体
\def\arrSeries#1#2{\Series{pink}{#1}{#2}}      % 箭头类型 , 红色底色 , 普通字体

\def\obj#1{\objSeries{10}{#1}}                 % 对象
\def\cat#1{\objSeries{30}{#1}}                 % 范畴

\def\arr#1{\arrSeries{90!Orchid}{#1}}          % 箭头
\def\fct#1{\arrSeries{10!Orchid}{#1}}          % 函子 
\def\ntf#1{\arrSeries{40!Orchid}{#1}}          % 自然变换

% 常用变量
\makeatletter
\def\varFreqUse#1#2{
  \expandafter\newcommand\expandafter{\csname #1#2\endcsname}[1][]%
  {\@ifnextchar p%
    {\csname #1#2@A\endcsname{##1}}%
    {\csname #1#2@B\endcsname{##1}}%
  }
  \expandafter\def\csname #1#2@A\endcsname##1p%
  {\@ifnextchar p%
    {\csname #1#2@A@A\endcsname{##1}p}%
    {\csname #1#2@A@B\endcsname{##1}p}%
  }
  \expandafter\def\csname #1#2@A@A\endcsname##1pp%
  {\csname #1\endcsname {#2_{##1}''}}
  \expandafter\def\csname #1#2@A@B\endcsname##1p%
  {\csname #1\endcsname {#2_{##1}'}}
  \expandafter\def\csname #1#2@B\endcsname##1%
  {\csname #1\endcsname {#2_{##1}}}
}
\varFreqUse{obj}{0}                            % 对象 0
\varFreqUse{obj}{1}                            % 对象 1
\varFreqUse{obj}{c}                            % 对象 c
\varFreqUse{obj}{d}                            % 对象 d
\varFreqUse{arr}{f}                            % 箭头 f
\varFreqUse{arr}{g}                            % 箭头 g
\varFreqUse{fct}{F}                            % 函子 F
\varFreqUse{fct}{G}                            % 函子 G
\varFreqUse{cat}{C}                            % 范畴 C
\varFreqUse{cat}{D}                            % 范畴 D
\varFreqUse{cat}{Cat}                          % 范畴 Cat
\varFreqUse{cat}{Set}                          % 范畴 Set
\makeatother
% 打印方法名
\newcommand{\opr}[2][\cat C]{                  % 方法的名称
  \plr\smash{\overset{\mclap{\small #1}}{#2}}
}
\def\rst#1undr#2{
  {}^{}_{\smash{:#2}}{\plr\smash{#1}}
}
\newcommand{\id}[1][\obj c]{
  \rst{\textrm{id}}undr#1
}