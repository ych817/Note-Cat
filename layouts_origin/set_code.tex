\usemintedstyle{material}                                    % 代码块主体高亮风格(见https://pygments.org/styles/)
\colorlet{colorCodefenceBodyDraw}{Gray}                      % 代码块主体边框颜色
\colorlet{colorCodefenceBodyFill}{Black!75!DarkBlue}         % 代码块主体填充颜色
\colorlet{colorCodefenceBodyFillHL}{black!45!DarkSlateBlue}  % 代码块主体填充颜色-高亮行

\colorlet{colorCodeinlineBodyDraw}{Gray}                     % 行内代码块边框颜色
\colorlet{colorCodeinlineBodyFill}{Black!75!DarkBlue}        % 行内代码块填充颜色
\colorlet{colorCodeinlineBodyFillHL}{black!45!DarkSlateBlue} % 行内代码块填充颜色-高亮行

\colorlet{colorCodefenceNmbrText}{white!25!LightSteelBlue}   % 代码块行号文字颜色
\colorlet{colorCodefenceNmbrFill}{colorCodefenceBodyFill}    % 代码块行号填充颜色

\colorlet{colorCodefenceTitlText}{Cerulean!50!white}         % 代码块行号文字颜色
\colorlet{colorCodefenceTitlDraw}{colorCodefenceNmbrText}    % 代码块标题边框颜色
\colorlet{colorCodefenceTitlFill}{colorCodefenceBodyFill}    % 代码块标题边框颜色

\tikzset{ % 用于下面 tcblisting 的 finish 键
  codeforce/title/.style={
    draw=colorCodefenceTitlDraw,
    fill=colorCodefenceTitlFill,
    text=colorCodefenceTitlText,
    line width=.4pt,
    font=\ttfamily\relscale{.7},
    rounded corners=3pt,
    outer sep=2pt
  }
}
\tcbset{
  codestyle/.style={
    fontupper=\relscale{.7},
    breakable,
    enhanced,
    skin first=enhanced,
    skin middle=enhanced,
    skin last=enhanced,
    colframe=colorCodefenceBodyDraw,
    colback =colorCodefenceBodyFill,
    boxrule=.4pt,
    arc=5pt,
  }
}
\setminted{
  breaklines=true,
  breakanywhere=true,
  escapeinside= 脎吡,%
  encoding=utf8,
}

% 使黑暗模式下所有代码都能可见 。代码来自 https://tex.stackexchange.com/a/742098/242598
\renewcommand\FancyVerbFormatText[1]{\textcolor{colorCodefenceNmbrText}{#1}}
\fvset{breaksymbolleft=\textcolor{colorCodefenceNmbrText}{\tiny\ensuremath{\hookrightarrow}}}
% 代码块中高亮行的行号也能被高亮 。代码来自 https://tex.stackexchange.com/a/741983/242598
% 相关的库 : minted , fbextra , fancyvrb , etoolbox
% 代码块中高亮行的行号也能被高亮 。代码来自 https://tex.stackexchange.com/a/741983/242598
% 相关的库 : minted , fbextra , fancyvrb , etoolbox
\makeatletter
\def\FancyVerbHighlightLineFirst#1{%
  \setlength{\FV@TmpLength}{\fboxsep}%
  \setlength{\fboxsep}{0pt}%
  %%% <<< begin patch 1
  \ifx\FV@LeftListNumber\relax
  \else
    \hspace{-\dimexpr\myFV@EstimatedNumberWidth+\FV@FrameRule+\FV@NumberSep}%
  \fi
  %%% >>> end patch 1
  \colorbox{\FancyVerbHighlightColor}{%
    \setlength{\fboxsep}{\FV@TmpLength}%
    %%% <<< begin patch 2
    \ifx\FV@LeftListNumber\relax
    \else
      \hspace{\myFV@EstimatedNumberWidth}%
      % typeset left line number again, because the original number is under
      % the line background
      \llap{\theFancyVerbLine}%
      \hspace{\dimexpr\FV@FrameRule+\FV@NumberSep}%
    \fi
    %%% >>> end patch 2
    \rlap{\strut#1}%
    \hspace{\linewidth}%
    \ifx\FV@RightListFrame\relax\else
      \hspace{-\FV@FrameSep}%
      \hspace{-\FV@FrameRule}%
    \fi
    \ifx\FV@LeftListFrame\relax\else
      \hspace{-\FV@FrameSep}%
      \hspace{-\FV@FrameRule}%
    \fi
  }%
  \hss
}
\let\FancyVerbHighlightLineMiddle\FancyVerbHighlightLineFirst
\let\FancyVerbHighlightLineLast\FancyVerbHighlightLineFirst
\let\FancyVerbHighlightLineSingle\FancyVerbHighlightLineFirst
\newlength{\myFV@EstimatedNumberWidth}
\setlength{\myFV@EstimatedNumberWidth}{18pt}
\makeatother
% 代码块行号格式设置
\renewcommand{\theFancyVerbLine}{%                         
    \ttfamily\textcolor{colorCodefenceNmbrText}{\relsize{-.5}%
        \oldstylenums{%
            \arabic{FancyVerbLine}%
        }%
    }%
}
% 代码块外观样式设置
% 相关的库 : minted , tcolorbox , xparse
\NewTCBListing{codefence}{ !o !E{\Syntax\Highlight\Insert\Delete}{{latex}{}{}{}} }{%
  codestyle,
  boxsep=2pt,
  right=-2pt, % 对应 boxsep
  % 代码块中标题和语法名称可以显示在 tcb 的上层 。代码来自 https://tex.stackexchange.com/a/742065/242598
  finish={
    \node(ext) [anchor=north east, codeforce/title]
      at (frame.north east) {#2};
    \IfValueT{#1}{
      \node(title) [anchor=east, codeforce/title]
        at (ext.west) {#1};
    }
  },
  listing only, 
  listing engine=minted,
  minted language = #2, 
  minted options = {   % minted 设置
    tabsize=2,
    linenos=true,
    numbersep=5pt,
    highlightlines={#3},
    highlightcolor=colorCodefenceBodyFillHL,
    xleftmargin=+10pt,
  }
}
% 行内代码块外观设置
% 相关的库 : tcolorbox
\newtcbox{\codeinlineTCB}{% 行间代码块设置
  codestyle,
  colupper=colorCodefenceNmbrText,
  nobeforeafter,
  boxsep=1.5pt,
  left=2pt,
  right=2pt,
  top=0pt,
  bottom=0pt,
  on line, % 位置设置
}
\newcommand{\codeinline}[2][latex]{%
  \vphantom{\scriptsize\ttfamily (Fg泥濠)}%
  \codeinlineTCB{\vphantom{\tt (Fg泥濠)}\mintinline{#1}{#2}}%
}
% 代码块中高亮一行内的部分内容 。
% 相关的库 : 
\newcommand{\codeHL}[2][colorCodefenceBodyFillHL]{%        % 代码高亮部分
    \setlength\fboxsep{0pt}%
    \rlap{\colorbox{#1}{\vphantom{(Fg)}\phantom{#2}}}%
    \setlength\fboxsep{3pt}%
}%