% keyval test
\makeatletter
\define@key{family}{color}[black]{\color{#1}}
\define@key{family}{font}{#1}
\def\mybox{%
  \@ifnextchar[% 如果下一个字符是 [ 
  \@mybox%     % 那么调用带参数的版本
  {\@mybox[]}% % 否则调用不带参数的版本
}
\def\@mybox[#1]#2{%
  \setkeys{family}{#1}%
  #2
  % some operations to typeset #2
}
\makeatother

% \mybox[%
%   color = red,
%   % font  = \ttfamily
% ]{%
%   This is my box%
% }%

\pgfkeys{
  /family/.is family,
  /family,
  color/.initial = black,
  color/.code    = \color{#1},
  font/.initial  = \normalfont,
  font/.code     = #1,
}
\renewcommand\mybox[2][]{%
  \pgfkeys{/family, #1}%
  #2
}

\pgfkeysvalueof{/family/color}

\mybox[%
  color = red,
  % font  = \ttfamily
]{%
  This is my box%
}%

\pgfkeysvalueof{/family/color}

% \def\helloworld{Hello, world!}
% \pgfkeyslet{/my family/my key}{\helloworld}
% \pgfkeysvalueof{/my family/my key}

% \pgfkeyssetvalue{/my family/my key}{Hello, world!}
% \pgfkeysgetvalue{/my family/my key}{\helloworld}
% \helloworld