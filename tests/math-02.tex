% 一系列别名
\let\ph=\phantom
\let\hph=\hphantom
\let\vph=\vphantom
\let\mrel=\mathrel
\let\mbin=\mathbin
\let\mllap=\mathllap
\let\mrlap=\mathrlap
\let\mclap=\mathclap

\def\vts#1{\lvert#1\rvert}                     % 竖线 ||
\def\prs#1{\left(#1\right)}                    % 括号 ()
\def\bcs#1{\left\{#1\right\}}                  % 括号 {}
\def\bks#1{\left[#1\right]}                    % 括号 []
\def\plr{\vph{(fg)}}                           % 柱子, 用于指定盒子的最小高度
\def\etc{\textrm{etc}}                         % 对应省略号
\def\wld{\_}                                   % 通配符

% 用于注册需要被高亮的数学对象
\def\RegistMathHL#1BG#2Font#3{
  \colorlet{color#1}{#2}
  \expandafter\newif\csname ifmathHLcolor#1\endcsname    % 确认是否开启背景高亮
  \expandafter\def\csname #1\endcsname##1{%
    \setlength{\fboxsep}{0pt}%
    \setlength{\fboxrule}{0pt}%
    \let\intermediate=\relax
    \csname ifmathHLcolor#1\endcsname{%
      \gdef\intermediate{\colorbox{color#1}}%
    }\else{%
      \gdef\intermediate{\fbox}%
    }\fi%
    \intermediate{\ensuremath{\plr\smash{#3 ##1}}}%
  }
  \csname mathHLcolor#1true\endcsname                    % 默认开启背景高亮
}

% 注册几种类型 : 对象 , 范畴 , 态射 , 函子 , 自然变换
\RegistMathHL{obj}BG{blue!10}Font{\sf}
\RegistMathHL{cat}BG{blue!30}Font{\sf}
\RegistMathHL{arr}BG{pink!90!Orchid}Font{}
\RegistMathHL{fct}BG{pink!10!Orchid}Font{}
\RegistMathHL{ntf}BG{pink!40!Orchid}Font{}

% 测试
$\obj a,\obj b,\obj c$ , $\cat C$ \par
$\arr f$ , $\ntf \eta$ , $\fct F$ \par

% 常用变量
\makeatletter
\def\RegistVarFreq#1#2{
  \expandafter\newcommand\expandafter{\csname #1#2\endcsname}[1][]%
  {\@ifnextchar p%
    {\csname #1#2@A\endcsname{##1}}%
    {\csname #1#2@B\endcsname{##1}}%
  }
  \expandafter\def\csname #1#2@A\endcsname##1p%
  {\@ifnextchar p%
    {\csname #1#2@A@A\endcsname{##1}p}%
    {\csname #1#2@A@B\endcsname{##1}p}%
  }
  \expandafter\def\csname #1#2@A@A\endcsname##1pp%
  {\csname #1\endcsname {#2_{##1}''}}
  \expandafter\def\csname #1#2@A@B\endcsname##1p%
  {\csname #1\endcsname {#2_{##1}'}}
  \expandafter\def\csname #1#2@B\endcsname##1%
  {\csname #1\endcsname {#2_{##1}}}
}
\makeatother

% 注册几个常用变量
\RegistVarFreq{obj}{0}                            % 对象 0
\RegistVarFreq{obj}{1}                            % 对象 1
\RegistVarFreq{obj}{c}                            % 对象 c
\RegistVarFreq{obj}{d}                            % 对象 d
\RegistVarFreq{arr}{f}                            % 箭头 f
\RegistVarFreq{arr}{g}                            % 箭头 g
\RegistVarFreq{fct}{F}                            % 函子 F
\RegistVarFreq{fct}{G}                            % 函子 G
\RegistVarFreq{cat}{C}                            % 范畴 C
\RegistVarFreq{cat}{D}                            % 范畴 D
\RegistVarFreq{cat}{Cat}                          % 范畴 Cat
\RegistVarFreq{cat}{Set}                          % 范畴 Set

% 测试
$\obj0$ , $\obj1$ , $\objc[1]pp$ , $\objd[1]p$ , $\objd[i]p$ \par
$\arrf$ , $\arrf[1]p$ , $\arrg[2]pp$ \par
$\fctF$ , $\fctG$

% 打印方法名
% \def\RegistMathOpr#1{%
%   \expandafter\def\csname #1\endcsname{%
%     \@ifnextchar [%
%       {\csname #1@A\endcsname}%
%       {\csname #1@B\endcsname}%
%   }%
%   \expandafter\def\csname #1@A\endcsname[##1]##2{%
%     Mandatory ##2 , Optional ##1%
%   }%
%   \expandafter\def\csname #1@B\endcsname##1{%
%     \csname #1@A\endcsname[Default]{##1}
%   }%
% }



\newcommand{\opr}[2][\catC]{                      % 方法的名称
  \plr\smash{\overset{\mclap{\small #1}}{#2}}
}

\makeatletter
\def\testOpt#1{
  \@ifnextchar <%
  {\testOpt@A#1}%
  {\testOpt@B#1}%
}
\def\testOpt@A#1<#2>{%
  Mandatory {\ttfamily #2} , Optional {\ttfamily #1}%
}%
\def\testOpt@B#1{%
  \testOpt@A#1[d]
}%
\makeatother

\testOpt{A}<x>
% \testOpt@A x

\makeatletter
\def\RegistMathOpr#1#2{
  \expandafter\def\csname #1\endcsname{%
    \@ifnextchar <%
      {\csname #1@A\endcsname}%
      {\csname #1@B\endcsname}%
  }%
  \expandafter\def\csname #1@A\endcsname<##1>{%
    \plr\smash{\overset{\mclap{\small ##1}}{#2}}
  }%
  \expandafter\def\csname #1@B\endcsname{%
    \csname #1@A\endcsname<\catC>
  }%
}
\makeatother

\RegistMathOpr{cattimes}{\times}

$\cattimes<\catD>$\par
$\cattimes$